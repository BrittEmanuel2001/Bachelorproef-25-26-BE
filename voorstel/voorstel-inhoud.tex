%---------- Inleiding ---------------------------------------------------------

\section{Inleiding}%
\label{sec:inleiding}

Het komt steeds vaker in het nieuws: jongvolwassenen die geconfronteerd worden met stress, angst of burn-out. Na hun middelbare school staan ze voor tal van belangrijke keuzes: verder studeren, aan het werk gaan, of kiezen voor een sabbatjaar. Bovendien valt het combineren van studies, werk en een eigen sociaal leven zeker niet te onderschatten. Dit leidt namelijk tot aanzienlijke druk op deze groep. Wat ook duidelijk wordt in het recente onderzoeksrapport van \textcite{RIZIV2023_invaliditeit_burnout}: het aantal burn-outs bij twintigers is sinds 2018 \textbf{meer dan verdubbeld}, terwijl de groei bij andere leeftijden beperkter bleef.

\vspace{1em}

Parallel verloopt alle contact vandaag de dag steeds meer via online platformen. Niet alleen voor het onderhouden van sociale contacten, maar ook voor het uitvoeren van werk- en studie gerelateerde activiteiten. Hoewel websites en apps veel voordelen bieden, zorgt deze continue digitale aanwezigheid ervoor dat de grens tussen werk, studie en ontspanning steeds meer vervaagt, wat op zijn beurt weer bijdraagt aan meer werkdruk en dus ook een groter risico op burn-out. Hierdoor is mentale gezondheid één van de meest urgente maatschappelijke uitdagingen voor Vlaamse twintigers.

\vspace{1em}

Om dit probleem aan te pakken hebben verscheidene software- en productontwikkelaars geprobeerd de situatie te verbeteren. Dit leidde tot een explosieve groei van meer dan \textbf{1.400 mentaal welzijn apps} die stress, angst en burn-out beweren tegen te gaan \autocite{Larsen2019}. Echter lijken deze een beperkt effect met zich mee te brengen. Dit roept dan ook verschillende vragen op over het huidige aanbod van mentaal welzijn apps: in hoeverre zijn ze wetenschappelijk onderbouwd, zijn ze afgestemd op de doelgroep en implementeren ze bovendien effectief de principes van ``\emph{calm design}'' zonder onnodige cognitieve belasting? Daarnaast, vanuit een develop-ment- en designperspectief, kan er zich ook worden afgevraagd wat exact valt onder ``\emph{calm design}``. Meer concreet: Welke interface elementen bevorderen mentale rust, en welke eerder tegen? Hoe dragen kleurgebruik, overzichtelijkheid, animaties, pushmeldingen en gamification bij binnen deze filosofie? Op basis van bovenstaande inzichten wordt volgende centrale onderzoeksvraag geformuleerd: 

\vspace{1em}

    \textit{``Hoe kunnen apps voor stress- en burn-out-preventie het mentale welzijn van Vlaamse twintigers verbeteren, door middel van inzichten uit een analyse van bestaande apps te vertalen naar concrete ontwerpprincipes en functionaliteiten voor een proof-of-concept app?''}

\vspace{1em}

Om de onderzoeksvraag te beantwoorden, werden de volgende deelvragen opgesteld, met speciale aandacht voor de doelgroep, bestaande apps en alternatieve initiatieven, app-functionaliteiten, interfaceontwerp, gebruiksvriendelijkheid, en ten slotte herbruikbaarheid.

\vspace{1em}
\textbf{\newline Deelvragen probleemdomein:}

\begin{enumerate}
    \item Welke categorieën bestaan er onder Belgische twintigers en wat zijn hun individuele specifieke noden op vlak van mentale gezondheid?
    \item Hoe beïnvloeden persoonlijke en externe factoren (werk-, studie- en sociale druk) de gebruikerservaring van digitale toepassingen, en welke functies en ondersteuning verwachten de verschillende subgroepen?
    
    \item Welke functies en ontwerpprincipes worden momenteel gebruikt in apps voor mentaal welzijn, en hoe effectief zijn deze?
    \item Welke UI/UX-elementen dragen bij aan mentale rust en welke verhogen juist cognitieve belasting?
    \item Waar falen bestaande apps in het ondersteunen van mentale gezondheid en het verminderen van stress of burn-out?
    \item Welke succesvolle ontwerpprincipes zijn herbruikbaar voor andere digitale toepassingen?
    
    \item Welke initiatieven, buiten apps, ondersteunen het mentale welzijn van twintigers, en welke best practices uit deze initiatieven kunnen worden vertaald naar digitale toepassingen?
\end{enumerate}

\textbf{\newline Deelvragen oplossingsdomein:}

\begin{enumerate}
    \item Welke functionele en niet-functionele vereisten zijn essentieel voor een app die stress en burn-out bij twintigers helpt voorkomen?
    
    \item Hoe kan de interface van een app ontworpen worden om mentale rust te bevorderen en cognitieve belasting te minimaliseren?
    \item Hoe kunnen meldingen en gamification zodanig worden ingezet dat ze ondersteunen zonder cognitief te overweldigen?
    
    \item Welke generieke ontwerpprincipes en functionaliteiten kunnen worden toegepast op andere digitale toepassingen voor mentale gezondheid?
    \item Hoe kunnen deze richtlijnen praktisch geïmplementeerd worden in andere apps?
\end{enumerate}

Op basis van deze onderzoeksvraag en deelvragen wordt de basis gevormd voor het analyseren van bestaande mentaal welzijn apps, het doelgericht identificeren van effectieve ontwerpprincipes, en uiteindelijk het ontwikkelen van een proof-of-concept app ter ondersteuning van burn-out-preventie bij Vlaamse twintigers. Het uiteindelijke doel is namelijk om hen een effectief, wetenschappelijk onderbouwd hulpmiddel te bieden ter preventie van burn-out, dat bovendien psychologen ondersteunt en hun werkdruk verlicht.

%---------- Stand van zaken ---------------------------------------------------

\section{Literatuurstudie}%
\label{sec:literatuurstudie}

\vspace{1em}

\subsection{Twintigers onder druk: \newline mentale gezondheid onder de loep}%
\label{subsec:twintigers_gezondheid}

\vspace{1em}

Volgens recente cijfers van \textcite{RIZIV2023_invaliditeit_burnout} is het aantal jongvolwassenen (< 30 jaar) met langdurige arbeidsongeschiktheid wegens burn-out tussen 2018 en 2023 gestegen met maar liefst 107\%, de sterkste toename van alle leeftijdsgroepen. De \textcite{SERV2024_werkbaarwerk} bevestigt daarbij de omvang van het probleem: 35,9\% van de jonge werknemers ervaart werkstress, en 13\% rapporteert burn-out-symptomen. 

\vspace{1em}

De oorzaak van deze verontrustende situatie bevindt zich zowel in werkcontext als uitdagingen die gepaard gaan binnen hun specifieke levensfase. Een analyse van \textcite{Veen2022} stelt dat hoge werkdruk, lage autonomie en gebrek aan steun of erkenning op het werk het risico op burn-out bij jonge werknemers het meest verhogen.

\vspace{1em}

Daarnaast speelt volgens \textcite{Hasyim2024} de \textbf{QLC (\emph{Quarter-Life Crisis})} een prominente rol. Zo heerst er heel wat onzekerheid in deze levensfase: jongvolwassenen die twijfelen over hun identiteit, toekomst, capaciteiten en hun doel in het leven (“\emph{purpose in life}”). Externe factoren zoals de noodzaak om financieel zelfstandig te worden en bezorgdheden rond carrièreontwikkeling versterken deze druk.

\vspace{1em}

\subsection{De rol van digitalisering in stress en cognitieve belasting}%
\label{subsec:digitalisering_stress}

\vspace{1em}

Sinds de COVID-pandemie nam de hoeveelheid aan telewerk aanzienlijk toe. Hoewel het volgens \textcite{Bondanini2025} zijn voordelen kan hebben, waaronder flexibiliteit en een betere \emph{work-life balance}, tonen \textcite{Costin2023} dat telewerk ook zijn negatieve effecten kent. Zo kan het leiden tot technologische vermoeidheid, sociaal isolement en psychologische belasting, welke op hun beurt productiviteit, betrokkenheid en werktevredenheid verminderen.

\vspace{1em}

Daarnaast speelt \textbf{technostress} ook een cruciale rol en gaat deze vaak hand in hand met burn-out: hoe hoger de ervaren stress, hoe groter de kans op burn-out \autocite{Consiglio2023}. Zo dragen techno-overload \emph{(sneller kunnen werken en dus bijgevolg meer)}, techno-invasion \emph{(vervaagde grens tussen werk- en privéleven)} en techno-com-plexity \emph{(continu moeten bijleren van nieuwe technologieën)} ieder hun steentje bij tot meer cognitieve overbelasting.

\vspace{1em}

\textbf{Online vigilence}, en dus een constante aandacht voor inkomende berichten en notificaties, draagt eveneens bij aan stress en een verhoogde cognitieve belasting \autocite{Freytag2020}. Hierbij vormen salience \emph{(het voortdurend aan online toepassingen en meldingen denken)}, reactibility \emph{(direct willen reageren)} en monitoring \emph{(frequent onnodige controle op nieuwe meldingen)} prominente oorzaken van deze verhoogde mentale belasting. Het gebruik van meerdere digitale platformen kan deze druk bovendien verder intensiveren, waardoor het mentale welzijn van de gebruiker alsmaar meer negatief wordt beïnvloed \autocite{Ibrahim2025}.

\vspace{1em}

\subsection{Een overzicht van huidige \newline burn-outpreventie- en \newline behandelingsmaatregelen}%
\label{subsec:preventiestrategieën}

\vspace{1em}
Burn-outpreventie en -behandeling gaan niet alleen over hoe het individu ermee omgaat, maar ook over hoe organisaties dit aanpakken. Op individueel niveau blijkt Rational Emotive Behavior Therapy (\textbf{REBT}), dat irrationele overtuigingen identificeert en herstructureert, het meest effectief \autocite{Madigan2023}. Ook \emph{psycho-educatie}, \emph{mindfulness} en \emph{cognitieve gedragstherapie} (\textbf{CBT}) kunnen bijdragen aan het verminderen van burn-out. Op organisatieniveau blijken volgens \textcite{Bes2023} vooral workload- en particitpatieve interventies effectief. Uit hun onderzoek blijkt een combinatie van individuele en organisatiegerichte strategieën het meest wenselijk en veelbelovend.
\vspace{1em}

\subsection{Effectiviteit van de huidige \newline mentale gezondheidsapps}%
\label{subsec:effectiviteit_huidige_apps}

\vspace{1em}

Vandaag de dag bestaan er veel mentale ge-zondheidsapps, 1435 om precies te zijn \autocite{Larsen2019}. Velen hiervan beweren effectief te zijn en ondersteunen dit met wetenschappelijke termen. Uit het onderzoek van \textcite{Larsen2019} bleek echter dat  er relatief weinig wetenschappelijk bewijs aanwezig is. Er bestaat dus een duidelijke kenniskloof tussen wat apps beloven en wat wetenschappelijk bewezen is.

\vspace{1em}

Desalniettemin, indien wetenschappelijk onderbouwd, laten bepaalde apps behoorlijk positieve effecten zien op het verminderen van stress en burn-out. Zo zagen \textcite{Deriglazov2025} dat mindfulness- en meditatieapps een significant positief effect kunnen hebben op gevoelens van professionele voldoening, waardoor ze zo kunnen bijdragen aan het verminderen van burn-outsymptomen. Ook \textcite{wojtyna2023mobile} publiceerden dat mobiele CBT-apps effectief kunnen zijn in het reduceren van werkstress en burn-out. Ze zijn even effectief als face-to-face CBT en vormen daarmee een kostenefficiënt en toegankelijk alternatief. Regulatie en het vastleggen van richtlijnen blijken hieruit essentieel om de kenniskloof te verkleinen bij ontwikkelaars.

\vspace{1em}

\subsection{Calm design in de theorie: \newline principes en toepassing binnen \newline productontwikkeling}%
\label{subsec:calm_design_principes}

\vspace{1em}

Het werd al duidelijk in voorgaande paragrafen dat het gebruik van digitale middelen het psychologisch welzijn van gebruikers kan verminderen. \emph{Calm design} biedt hier een antwoord op door digitale producten niet alleen functioneel te maken, maar ook bij te dragen aan mentale rust en het verminderen van cognitieve belasting \autocite{Peters2022}. Op basis van de \textbf{Self-Determination Theory} identificeerde \textcite{Peters2022} drie basispsychologische behoeften die de kern vormen van calm design: \emph{autonomy}, \emph{competence} en \emph{relatedness}. Concreet kan dit zich vertalen in ontwerpkeuzes zoals het beperken van interrupties, ondersteunen van focusmomenten of faciliteren van sociale interactie. Daarnaast benadrukt onderzoek dat digitale frictie, duidelijke feedback en sociale samenwerking kunnen helpen impulsief gebruik te beperken en betekenisvol gebruik te stimuleren \autocite{Almoallim2022}.

\vspace{1em}

Het bestaan van calm design principes zal zeker hun meerwaarde kunnen leveren binnen dit onderzoek, maar er is nog relatief weinig bekend over hoe specifieke interface-elementen precies bijdragen aan mentale rust ter ondersteuning van het behandelen van burn-outs. Toch vormen deze principes een nuttige leidraad en kunnen ze als basis dienen bij de ontwikkeling van de proof-of-concept app.

%---------- Methodologie ------------------------------------------------------

\section{Methodologie}%
\label{sec:methodologie}

Hier beschrijf je hoe je van plan bent het onderzoek te voeren. Welke onderzoekstechniek ga je toepassen om elk van je onderzoeksvragen te beantwoorden? Gebruik je hiervoor literatuurstudie, interviews met belanghebbenden (bv.~voor requirements-analyse), experimenten, simulaties, vergelijkende studie, risico-analyse, PoC, \ldots?

Valt je onderwerp onder één van de typische soorten bachelorproeven die besproken zijn in de lessen Research Methods (bv.\ vergelijkende studie of risico-analyse)? Zorg er dan ook voor dat we duidelijk de verschillende stappen terug vinden die we verwachten in dit soort onderzoek!

Vermijd onderzoekstechnieken die geen objectieve, meetbare resultaten kunnen opleveren. Enquêtes, bijvoorbeeld, zijn voor een bachelorproef informatica meestal \textbf{niet geschikt}. De antwoorden zijn eerder meningen dan feiten en in de praktijk blijkt het ook bijzonder moeilijk om voldoende respondenten te vinden. Studenten die een enquête willen voeren, hebben meestal ook geen goede definitie van de populatie, waardoor ook niet kan aangetoond worden dat eventuele resultaten representatief zijn.

Uit dit onderdeel moet duidelijk naar voor komen dat je bachelorproef ook technisch voldoen\-de diepgang zal bevatten. Het zou niet kloppen als een bachelorproef informatica ook door bv.\ een student marketing zou kunnen uitgevoerd worden.

Je beschrijft ook al welke tools (hardware, software, diensten, \ldots) je denkt hiervoor te gebruiken of te ontwikkelen.

Probeer ook een tijdschatting te maken. Hoe lang zal je met elke fase van je onderzoek bezig zijn en wat zijn de concrete \emph{deliverables} in elke fase?

%---------- Verwachte resultaten ----------------------------------------------

\section{Verwacht resultaat, conclusie}%
\label{sec:verwachte_resultaten}

Hier beschrijf je welke resultaten je verwacht. Als je metingen en simulaties uitvoert, kan je hier al mock-ups maken van de grafieken samen met de verwachte conclusies. Benoem zeker al je assen en de onderdelen van de grafiek die je gaat gebruiken. Dit zorgt ervoor dat je concreet weet welk soort data je moet verzamelen en hoe je die moet meten.

Wat heeft de doelgroep van je onderzoek aan het resultaat? Op welke manier zorgt jouw bachelorproef voor een meerwaarde?

Hier beschrijf je wat je verwacht uit je onderzoek, met de motivatie waarom. Het is \textbf{niet} erg indien uit je onderzoek andere resultaten en conclusies vloeien dan dat je hier beschrijft: het is dan juist interessant om te onderzoeken waarom jouw hypothesen niet overeenkomen met de resultaten.


%---------- Inleiding ---------------------------------------------------------

\section{Inleiding}%
\label{sec:inleiding}

Het komt steeds vaker in het nieuws: jongvolwassenen die geconfronteerd worden met stress, angst of burn-out. Na hun middelbare school staan ze voor tal van belangrijke keuzes: verder studeren, aan het werk gaan, of kiezen voor een sabbatjaar. Bovendien valt het combineren van studies, werk en een eigen sociaal leven zeker niet te onderschatten. Dit leidt namelijk tot aanzienlijke druk op deze groep. Wat ook duidelijk wordt in het recente onderzoeksrapport van \textcite{RIZIV2023_invaliditeit_burnout}: het aantal burn-outs bij twintigers is sinds 2018 \textbf{meer dan verdubbeld}, terwijl de stijging in andere leeftijdsgroepen veel minder groot is.

\vspace{1em}

Parallel verloopt alle contact vandaag de dag steeds meer via online platformen. Niet alleen voor het onderhouden van sociale contacten, maar ook voor het uitvoeren van werk- en studie gerelateerde activiteiten. Hoewel websites en apps veel voordelen bieden, zorgt deze continue digitale aanwezigheid ervoor dat de grens tussen werk, studie en ontspanning steeds meer vervaagt, wat op zijn beurt weer bijdraagt aan meer werkdruk en dus ook een groter risico op burn-out. 

\vspace{1em}

Hierdoor is mentale gezondheid één van de meest urgente maatschappelijke uitdagingen voor Vlaamse twintigers.

\vspace{1em}

Om dit probleem aan te pakken hebben verscheidene software- en productontwikkelaars geprobeerd de situatie te verbeteren. Dit leidde tot een explosieve groei van meer dan \textbf{1.400 mentaal welzijn apps} die stress, angst en burn-out beweren tegen te gaan \autocite{Larsen2019}.
Echter lijken deze een beperkt effect met zich mee te brengen. Dit roept dan ook verschillende vragen op over het huidige aanbod van mentaal welzijn apps: in hoeverre zijn ze wetenschappelijk onderbouwd, zijn ze afgestemd op de doelgroep en implementeren ze bovendien effectief de principes van ``\emph{calm design}'' zonder onnodige cognitieve belasting? Daarnaast, vanuit een develop-ment- en designperspectief, kan er zich ook worden afgevraagd wat exact valt onder ``\emph{calm design}``. Meer concreet: Welke interface elementen bevorderen mentale rust, en welke eerder tegen? Hoe dragen kleurgebruik, overzichtelijkheid, animaties, pushmeldingen en gamification bij binnen deze filosofie? Op basis van bovenstaande inzichten wordt volgende centrale onderzoeksvraag geformuleerd: 

\vspace{1em}

    \textit{``Hoe kunnen apps voor stress- en burn-out-preventie het mentale welzijn van Vlaamse twintigers verbeteren, door middel van inzichten uit een analyse van bestaande apps te vertalen naar concrete ontwerpprincipes en functionaliteiten voor een proof-of-concept app?''}

\vspace{1em}

Om de onderzoeksvraag te beantwoorden, werden de volgende deelvragen opgesteld, met speciale aandacht voor de doelgroep, bestaande apps en alternatieve initiatieven, app-functionaliteiten, interfaceontwerp, gebruiksvriendelijkheid, en ten slotte herbruikbaarheid.

\vspace{1em}
\textbf{\newline Deelvragen probleemdomein:}

\begin{enumerate}
    \item Welke categorieën bestaan er onder Belgische twintigers en wat zijn hun individuele specifieke noden op vlak van mentale gezondheid?
    \item Hoe beïnvloeden persoonlijke en externe factoren (werk-, studie- en sociale druk) de gebruikerservaring van digitale toepassingen, en welke functies en ondersteuning verwachten de verschillende subgroepen?
    
    \item Welke functies en ontwerpprincipes worden momenteel gebruikt in apps voor mentaal welzijn, en hoe effectief zijn deze?
    \item Welke UI/UX-elementen dragen bij aan mentale rust en welke verhogen juist cognitieve belasting?
    \item Waar falen bestaande apps in het ondersteunen van mentale gezondheid en het verminderen van stress of burn-out?
    \item Welke succesvolle ontwerpprincipes zijn herbruikbaar voor andere digitale toepassingen?
    
    \item Welke initiatieven, buiten apps, ondersteunen het mentale welzijn van twintigers, en welke best practices uit deze initiatieven kunnen worden vertaald naar digitale toepassingen?
\end{enumerate}

\textbf{\newline Deelvragen oplossingsdomein:}

\begin{enumerate}
    \item Welke functionele en niet-functionele vereisten zijn essentieel voor een app die stress en burn-out bij twintigers helpt voorkomen?
    
    \item Hoe kan de interface van een app ontworpen worden om mentale rust te bevorderen en cognitieve belasting te minimaliseren?
    \item Hoe kunnen meldingen en gamification zodanig worden ingezet dat ze ondersteunen zonder cognitief te overweldigen?
    
    \item Welke generieke ontwerpprincipes en functionaliteiten kunnen worden toegepast op andere digitale toepassingen voor mentale gezondheid?
    \item Hoe kunnen deze richtlijnen praktisch geïmplementeerd worden in andere apps?
\end{enumerate}

Op basis van deze onderzoeksvraag en deelvragen wordt de basis gevormd voor het analyseren van bestaande mentaal welzijn apps, het doelgericht identificeren van effectieve ontwerpprincipes, en uiteindelijk het ontwikkelen van een proof-of-concept app ter ondersteuning van burn-out-preventie bij Vlaamse twintigers. Het uiteindelijke doel is namelijk om hen een effectief, wetenschappelijk onderbouwd hulpmiddel te bieden ter preventie van burn-out, dat bovendien psychologen ondersteunt en hun werkdruk verlicht.

%---------- Stand van zaken ---------------------------------------------------

\section{Literatuurstudie}%
\label{sec:literatuurstudie}

Hier beschrijf je de \emph{state-of-the-art} rondom je gekozen onderzoeksdomein, d.w.z.\ een inleidende, doorlopende tekst over het onderzoeksdomein van je bachelorproef. Je steunt daarbij heel sterk op de professionele \emph{vakliteratuur}, en niet zozeer op populariserende teksten voor een breed publiek. Wat is de huidige stand van zaken in dit domein, en wat zijn nog eventuele open vragen (die misschien de aanleiding waren tot je onderzoeksvraag!)?

Je mag de titel van deze sectie ook aanpassen (literatuurstudie, stand van zaken, enz.). Zijn er al gelijkaardige onderzoeken gevoerd? Wat concluderen ze? Wat is het verschil met jouw onderzoek?

%Verwijs bij elke introductie van een term of bewering over het domein naar de vakliteratuur, bijvoorbeeld~\autocite{Hykes2013}! Denk zeker goed na welke werken je refereert en waarom.

Draag zorg voor correcte literatuurverwijzingen! Een bronvermelding hoort thuis \emph{binnen} de zin waar je je op die bron baseert, dus niet er buiten! Maak meteen een verwijzing als je gebruik maakt van een bron. Doe dit dus \emph{niet} aan het einde van een lange paragraaf. Baseer nooit teveel aansluitende tekst op eenzelfde bron.

Als je informatie over bronnen verzamelt in JabRef, zorg er dan voor dat alle nodige info aanwezig is om de bron terug te vinden (zoals uitvoerig besproken in de lessen Research Methods).

%Een testje om te zien hoe die autocite werkt zonder tilde... \autocite{Hasyim2024}
%
%Een testje om te zien hoe die autocite werkt met tilde~\autocite{Hasyim2024}
%
%Een testje om te zien hoe autocite werkt binnen een zin volgens \textcite{Hasyim2024}.

% Voor literatuurverwijzingen zijn er twee belangrijke commando's:
% \autocite{KEY} => (Auteur, jaartal) Gebruik dit als de naam van de auteur
%   geen onderdeel is van de zin.
% \textcite{KEY} => Auteur (jaartal)  Gebruik dit als de auteursnaam wel een
%   functie heeft in de zin (bv. ``Uit onderzoek door Doll & Hill (1954) bleek
%   ...'')

Je mag deze sectie nog verder onderverdelen in subsecties als dit de structuur van de tekst kan verduidelijken.

%---------- Methodologie ------------------------------------------------------

\section{Methodologie}%
\label{sec:methodologie}

Hier beschrijf je hoe je van plan bent het onderzoek te voeren. Welke onderzoekstechniek ga je toepassen om elk van je onderzoeksvragen te beantwoorden? Gebruik je hiervoor literatuurstudie, interviews met belanghebbenden (bv.~voor requirements-analyse), experimenten, simulaties, vergelijkende studie, risico-analyse, PoC, \ldots?

Valt je onderwerp onder één van de typische soorten bachelorproeven die besproken zijn in de lessen Research Methods (bv.\ vergelijkende studie of risico-analyse)? Zorg er dan ook voor dat we duidelijk de verschillende stappen terug vinden die we verwachten in dit soort onderzoek!

Vermijd onderzoekstechnieken die geen objectieve, meetbare resultaten kunnen opleveren. Enquêtes, bijvoorbeeld, zijn voor een bachelorproef informatica meestal \textbf{niet geschikt}. De antwoorden zijn eerder meningen dan feiten en in de praktijk blijkt het ook bijzonder moeilijk om voldoende respondenten te vinden. Studenten die een enquête willen voeren, hebben meestal ook geen goede definitie van de populatie, waardoor ook niet kan aangetoond worden dat eventuele resultaten representatief zijn.

Uit dit onderdeel moet duidelijk naar voor komen dat je bachelorproef ook technisch voldoen\-de diepgang zal bevatten. Het zou niet kloppen als een bachelorproef informatica ook door bv.\ een student marketing zou kunnen uitgevoerd worden.

Je beschrijft ook al welke tools (hardware, software, diensten, \ldots) je denkt hiervoor te gebruiken of te ontwikkelen.

Probeer ook een tijdschatting te maken. Hoe lang zal je met elke fase van je onderzoek bezig zijn en wat zijn de concrete \emph{deliverables} in elke fase?

%---------- Verwachte resultaten ----------------------------------------------

\section{Verwacht resultaat, conclusie}%
\label{sec:verwachte_resultaten}

Hier beschrijf je welke resultaten je verwacht. Als je metingen en simulaties uitvoert, kan je hier al mock-ups maken van de grafieken samen met de verwachte conclusies. Benoem zeker al je assen en de onderdelen van de grafiek die je gaat gebruiken. Dit zorgt ervoor dat je concreet weet welk soort data je moet verzamelen en hoe je die moet meten.

Wat heeft de doelgroep van je onderzoek aan het resultaat? Op welke manier zorgt jouw bachelorproef voor een meerwaarde?

Hier beschrijf je wat je verwacht uit je onderzoek, met de motivatie waarom. Het is \textbf{niet} erg indien uit je onderzoek andere resultaten en conclusies vloeien dan dat je hier beschrijft: het is dan juist interessant om te onderzoeken waarom jouw hypothesen niet overeenkomen met de resultaten.

